\documentclass[a4paper]{article}
\usepackage[affil-it]{authblk}
\usepackage[backend=bibtex,style=numeric]{biblatex}

\usepackage{amsmath,amssymb, geometry}
\geometry{margin=1.5cm, vmargin={0pt,1cm}}
\setlength{\topmargin}{-1cm}
\setlength{\paperheight}{29.7cm}
\setlength{\textheight}{25.3cm}

\addbibresource{citation.bib}

\begin{document}
% =================================================
\title{Numerical Analysis homework 1}

\author[1]{Jerry Xu 3230101329}
\affil[1]{Mathematics and Applied Mathematics (Strengthening Basic Science Program),class 2301, Zhejiang University}
\affil[1]{\texttt{3230101329@zju.edu.cn}}


\date{Due time: \today}

\maketitle

\begin{abstract}
    The abstract is not necessary for the theoretical homework, 
    but for the programming project, 
    you are encouraged to write one.      
\end{abstract}





% ============================================
\section*{I. The content of the homework}

\subsection*{question 1.8.1 I - VIII,  Theoretical Questions, on page 7}
\noindent\rule{\textwidth}{.4pt}

\subsubsection*{I.}
Consider the bisection method starting with the initial interval $[a, b] = [1.5, 3.5]$. In the following questions ``the interval'' refers to the bisection interval whose width changes across different loops.
    \begin{itemize}
        \item What is the width of the interval at the $n$th step?
        \item What is the supremum of the distance between the root $r$ and the midpoint of the interval?
    \end{itemize}

\textbf{Solution.}

Let the initial interval be $[a,b]=[1.5,3.5]$, so $b-a=2$.


Each bisection halves the interval width. If we measure the step count so that after $n$ bisections the interval has been halved $n$ times, the width at the $n$-th step is
\[
w_n=\frac{b-a}{2^{\,n}}.
\]
With $b-a=2$ this becomes
\[
\boxed{\,w_n=\frac{2}{2^{\,n}}=2^{\,1-n}\,.}
\]


Let $m_n$ be the midpoint of the interval at the $n$-th step. For any root \(r\) that is in the current interval, the distance \(|r-m_n|\) is at most half the interval width (because the midpoint splits the interval). Thus
\[
|r-m_n|\le \frac{w_n}{2}=\frac{b-a}{2^{\,n+1}}.
\]
Therefore the supremum (in all possible locations of \(r\) within the interval) is
\[
\boxed{\sup |r-m_n|=\frac{b-a}{2^{\,n+1}}.}
\]
With $b-a=2$ this simplifies to
\[
\boxed{\sup |r-m_n|=\frac{2}{2^{\,n+1}}=2^{-n}.}
\]





\noindent\rule{\textwidth}{.4pt}


\subsubsection*{II.}
In using the bisection algorithm with its initial interval as $[a_0, b_0]$ with $a_0 > 0$, we want to determine the root with its relative error no greater than $\epsilon$. Prove that this goal of convergence is achieved with the following choice of the number of steps,
    \[
    n \ge \frac{\log(b_0 - a_0) - \log\epsilon - \log a_0}{\log 2} - 1.
    \]

\textbf{Solution.}

We use the same notation as before. Let $r$ be the root contained in the initial interval $[a_0,b_0]$ with $a_0>0$, and let $m_n$ denote the midpoint after $n$ bisection steps. From the bisection property we have the midpoint error bound
\[
|r-m_n|\le \frac{b_0-a_0}{2^{\,n+1}}.
\]

We require the relative error to satisfy
\[
\frac{|r-m_n|}{|r|}\le \epsilon.
\]
Since $r\ge a_0>0$, a sufficient condition is
\[
\frac{|r-m_n|}{a_0}\le \epsilon,
\]
and using precvious inequation of midpoint error bound this becomes
\[
\frac{b_0-a_0}{2^{\,n+1}a_0}\le \epsilon.
\]
then we have:
\[
2^{\,n+1}\ge \frac{b_0-a_0}{a_0\epsilon}.
\]
Taking logarithms (base $2$) gives
\[
n+1 \ge \log_2\!\left(\frac{b_0-a_0}{a_0\epsilon}\right)
= \frac{\log(b_0-a_0)-\log a_0 - \log\epsilon}{\log 2}.
\]
Hence
\[
n \ge \frac{\log(b_0-a_0)-\log a_0 - \log\epsilon}{\log 2} - 1,
\]
which is the claimed bound. 


\noindent\rule{\textwidth}{.4pt}

\subsubsection*{III.}
Perform four iterations of Newton's method for the polynomial equation $p(x) = 4x^3 - 2x^2 + 3 = 0$ with the starting point $x_0 = -1$. Use a hand calculator and organize results of the iterations in a table.

\textbf{Solution.}

Consider
\[
p(x)=4x^3-2x^2+3,\qquad p'(x)=12x^2-4x.
\]
Newton's iteration is
\[
x_{n+1}=x_n-\frac{p(x_n)}{p'(x_n)}.
\]

We perform four iterations starting from \(x_0=-1\).  Below each step shows the values of \(p(x_n)\), \(p'(x_n)\) and the next iterate.

\[
\begin{aligned}
x_0 &= -1,\\
p(x_0) &= 4(-1)^3-2(-1)^2+3 = -4-2+3 = -3,\\
p'(x_0) &= 12(-1)^2-4(-1)=16,\\
x_1 &= x_0-\frac{p(x_0)}{p'(x_0)} = -1 -\frac{-3}{16} = -1+\frac{3}{16} = -\tfrac{13}{16} = -0.8125.
\end{aligned}
\]

\[
\begin{aligned}
x_1 &\approx -0.812500000000,\\
p(x_1) &\approx -0.465820312500,\\
p'(x_1) &\approx 11.171875000000,\\
x_2 &\approx x_1-\frac{p(x_1)}{p'(x_1)} \approx -0.770804195804.
\end{aligned}
\]

\[
\begin{aligned}
x_2 &\approx -0.770804195804,\\
p(x_2) &\approx -0.020137886720,\\
p'(x_2) &\approx 10.212886082449,\\
x_3 &\approx x_2-\frac{p(x_2)}{p'(x_2)} \approx -0.768832384256.
\end{aligned}
\]

\[
\begin{aligned}
x_3 &\approx -0.768832384256,\\
p(x_3) &\approx -0.000043708433,\\
p'(x_3) &\approx 10.168568357988,\\
x_4 &\approx x_3-\frac{p(x_3)}{p'(x_3)} \approx -0.768828085870.
\end{aligned}
\]

\[
\begin{aligned}
x_4 &\approx -0.768828085870,\\
p(x_4) &\approx -2.0741\times 10^{-10},\\
p'(x_4) &\approx 10.168471850942.
\end{aligned}
\]

For clarity, the iteration data (rounded for a hand calculator) are tabulated below:

\begin{center}
\begin{tabular}{c|c|c|c|c}
\hline
$n$ & $x_n$ & $p(x_n)$ & $p'(x_n)$ & $x_{n+1}$ \\
\hline
0 & $-1.000000000000$ & $-3.000000000000$ & $16.000000000000$ & $-0.812500000000$ \\
1 & $-0.812500000000$ & $-0.465820312500$ & $11.171875000000$ & $-0.770804195804$ \\
2 & $-0.770804195804$ & $-0.020137886720$ & $10.212886082449$ & $-0.768832384256$ \\
3 & $-0.768832384256$ & $-0.000043708433$ & $10.168568357988$ & $-0.768828085870$ \\
4 & $-0.768828085870$ & $-2.0741\times10^{-10}$ & $10.168471850942$ & --- \\
\hline
\end{tabular}
\end{center}

After four Newton iterations we obtain
\[
x_4 \approx -0.768828085870,
\]
with residual \(p(x_4)\approx -2.07\times 10^{-10}\), indicating good convergence.




\noindent\rule{\textwidth}{.4pt}

\subsubsection*{IV.}
Consider a variation of Newton's method in which only the derivative at $x_0$ is used,
\[
x_{n+1} = x_n - \frac{f(x_n)}{f'(x_0)}.
\]
Find $C$ and $s$ such that
\[
e_{n+1} = C e_n^s,
\]
where $e_n$ is the error of Newton's method at step $n$, $s$ is a constant, and $C$ may depend on $x_n$, the true solution $\alpha$, and the derivative of the function $f$.

\textbf{Solution.}

Let $\alpha$ be the true root, and denote the error by $e_n = x_n-\alpha$.
We use Taylor expansion of $f$ about $\alpha$:
\[
f(x_n) = f(\alpha + e_n) = f'(\alpha)e_n + \tfrac{1}{2}f''(\alpha)e_n^2 + O(e_n^3).
\]
Also write the denominator (the fixed derivative at $x_0$) as
\[
f'(x_0) = f'(\alpha) + \big(f'(x_0)-f'(\alpha)\big).
\]
For brevity set
\[
\Delta := f'(x_0)-f'(\alpha).
\]

The iteration is
\[
e_{n+1} = x_{n+1}-\alpha
= x_n - \frac{f(x_n)}{f'(x_0)} - \alpha
= e_n - \frac{f(x_n)}{f'(x_0)}.
\]
Substitute the Taylor series for $f(x_n)$:
\[
e_{n+1}
= e_n - \frac{f'(\alpha)e_n + \tfrac{1}{2}f''(\alpha)e_n^2 + O(e_n^3)}
{f'(\alpha)+\Delta}.
\]

Factor out $e_n$ and simplify the leading terms:
\[
\begin{aligned}
e_{n+1}
&= e_n\Big(1-\frac{f'(\alpha)}{f'(\alpha)+\Delta}\Big)
- \frac{\tfrac{1}{2}f''(\alpha)e_n^2}{f'(\alpha)+\Delta} + O(e_n^3)\\[4pt]
&= e_n\cdot\frac{\Delta}{f'(\alpha)+\Delta}
- \frac{1}{2}\frac{f''(\alpha)}{f'(\alpha)+\Delta}\,e_n^2 + O(e_n^3).
\end{aligned}
\]

From this expansion we see the following:

\begin{itemize}
  \item \textbf{Generic case:} If $\Delta \neq 0$ (i.e. $f'(x_0)\neq f'(\alpha)$),
  the linear term is dominant. Thus
  \[
  e_{n+1} = C\, e_n^1 + \text{(higher order terms)},
  \]
  with
  \[
  \boxed{\,s = 1,\qquad
  C = \frac{\Delta}{f'(x_0)} = \frac{f'(x_0)-f'(\alpha)}{f'(x_0)}.}
  \]
  (Here we used $f'(x_0)=f'(\alpha)+\Delta$ to write $C$ in this simple form.)
  So the method is generically \emph{linearly} convergent with factor $C$.
  
  \item \textbf{Special case:} If $\Delta = 0$ (i.e. $f'(x_0)=f'(\alpha)$),
  then the linear coefficient vanishes and the next term is quadratic. In this
  special case
  \[
  e_{n+1} = -\frac{1}{2}\frac{f''(\alpha)}{f'(\alpha)}\,e_n^2 + O(e_n^3),
  \]
  so
  \[
  \boxed{\,s = 2,\qquad
  C = -\frac{1}{2}\frac{f''(\alpha)}{f'(\alpha)}.}
  \]
  That is, if the fixed derivative at $x_0$ happens to equal $f'(\alpha)$,
  then the method has quadratic leading behaviour (like usual Newton).
\end{itemize}

In short, generically \(s=1\) with \(C=(f'(x_0)-f'(\alpha))/f'(x_0)\). Only when
\(f'(x_0)=f'(\alpha)\) the linear term disappears and we get \(s=2\) with
\(C=-\tfrac{1}{2}f''(\alpha)/f'(\alpha)\).




\noindent\rule{\textwidth}{.4pt}

\subsubsection*{V.}
Within $\left(-\frac{\pi}{2}, \frac{\pi}{2}\right)$, will the iteration $x_{n+1} = \tan^{-1} x_n$ converge?

\textbf{Solution.}

Consider the iteration \(x_{n+1}=\arctan x_n\) with \(x_0\in\big(-\tfrac{\pi}{2},\tfrac{\pi}{2}\big)\).

\textbf{Case 1:} \(x_0>0\). Since \(\tan y>y\) for \(y\in(0,\tfrac{\pi}{2})\), we have
\[
\arctan x < x \quad\text{for all } x>0.
\]
Hence \(x_1=\arctan x_0<x_0\). By induction, if \(x_n>0\) then \(x_{n+1}=\arctan x_n<x_n\), so the sequence \((x_n)\) is positive and monotone decreasing. Also each \(x_n\ge 0\) (clear from \(\arctan x\ge 0\) when \(x\ge0\)). A positive monotone decreasing sequence is bounded below and therefore converges.

\textbf{Case 2:} \(x_0<0\). Note \(\arctan\) is an odd function, so the same argument works: the sequence is negative and monotone increasing, hence it also converges.

\textbf{Case 3:} \(x_0=0\). Then \(x_n\equiv0\) and it converges.

In all cases the limit must satisfy \(L=\arctan L\), so \(L=0\). Thus the iteration converges for any initial \(x_0\in\big(-\tfrac{\pi}{2},\tfrac{\pi}{2}\big)\), and the limit is \(0\).



\noindent\rule{\textwidth}{.4pt}

\subsubsection*{VI.}

Let $p > 1$. What is the value of the following continued fraction?
\[
x = \frac{1}{p + \frac{1}{p + \frac{1}{p+ \cdots}}}
\]
Prove that the sequence of values converges. (Hint: this can be interpreted as $x = \lim_{n \to \infty} x_n$, where $x_1 = \frac{1}{p}$, $x_2 = \frac{1}{p + \frac{1}{p}}$, $x_3 = \frac{1}{p + \frac{1}{p + \frac{1}{p}}}$, and so forth. Formulate $x$ as a fixed point of some function.)

\textbf{Solution.}

Let \(p>1\) and define the finite-level values by
\[
x_1=\frac{1}{p},\qquad x_{n+1}=\frac{1}{p+x_n}\quad(n\ge1).
\]
We show \((x_n)\) converges by using monotone bounded subsequences.

1. \emph{Basic bounds.}

Since \(p>1\) we have \(x_1=1/p>0\). For any \(x\ge0\),
\(\dfrac{1}{p+x}\le\dfrac{1}{p}\). Thus every term satisfies
\[
0<x_n\le\frac{1}{p}\qquad\text{for all }n.
\]
So the sequence is positive and bounded.

2. \emph{Monotonicity of even and odd subsequences.}

The function \(f(x)=\dfrac{1}{p+x}\) is strictly decreasing on \([0,\infty)\).
Hence the composition \(f\circ f\) is strictly increasing on \([0,\infty)\).
But
\[
x_{n+2}=f(f(x_n))\quad\text{for all }n,
\]
so the subsequence \((x_{2n})\) (even terms) is generated by iterating the
increasing map \(f\circ f\). In particular, one checks by induction that
\((x_{2n})\) is monotone increasing, and similarly \((x_{2n+1})\) (odd terms)
is monotone decreasing. More concretely:
\[
x_2=f(x_1)\le x_1,\qquad x_4=f(f(x_2))\ge x_2,\ \ldots
\]
so \(x_2\le x_4\le x_6\le\cdots\) and \(x_1\ge x_3\ge x_5\ge\cdots\).

3. \emph{Convergence of subsequences and same limit.}

Both subsequences are monotone and bounded (by step 1), so they converge:
there exist limits \(L_{\text{even}}\) and \(L_{\text{odd}}\). Passing to the
limit in the relation \(x_{n+1}=f(x_n)\) along even and odd indices shows
these two limits must be fixed points of \(f\circ f\). But any fixed point
of \(f\circ f\) that lies in \([0,1/p]\) is also a fixed point of \(f\)
itself (because if \(y=f(f(y))\) then applying \(f\) to both sides gives
\(f(y)=f(f(f(y)))\), and by uniqueness one gets \(y=f(y)\); alternately one
can argue the two subsequence limits must be equal by continuity). Hence
\(L_{\text{even}}=L_{\text{odd}}=:a\). Therefore the whole sequence \(x_n\)
converges to the same limit \(a\).

4. \emph{Value of the limit.}

Taking limit \(n\to\infty\) in \(x_{n+1}=\dfrac{1}{p+x_n}\) gives
\[
a=\frac{1}{p+a}.
\]
Multiply both sides by \(p+a\) and rearrange:
\[
a(p+a)=1 \quad\Longrightarrow\quad a^2 + p a -1 =0.
\]
Solve the quadratic:
\[
a=\frac{-p\pm\sqrt{p^2+4}}{2}.
\]
Since \(a>0\), we take the positive root. Thus
\[
\boxed{\,a=\frac{-p+\sqrt{p^2+4}}{2}\,}
\]
is the value of the continued fraction. This completes the proof that the
sequence converges and gives the limit.



\noindent\rule{\textwidth}{.4pt}

\subsubsection*{VII.}
What happens in problem II if $a_0 < 0 < b_0$? Derive an inequality of the number of steps similar to that in II. In this case, is the relative error still an appropriate measure?


\textbf{Solution.}

We keep the same notation as in problem II. Now suppose the initial
interval satisfies \(a_0<0<b_0\). Let \(r\) be the root in \([a_0,b_0]\),
and let \(m_n\) be the midpoint after \(n\) bisections. As before the
width of the interval after \(n\) steps is
\[
w_n=\frac{b_0-a_0}{2^{\,n}},
\]
and the worst-case distance from the midpoint to the root is
\[
|r-m_n|\le \frac{w_n}{2}=\frac{b_0-a_0}{2^{\,n+1}}.
\]

1. \emph{Absolute-error requirement.} 

If we want the absolute error to be at most
\(\eta>0\), i.e.
\[
|r-m_n|\le \eta,
\]
then it is sufficient to require
\[
\frac{b_0-a_0}{2^{\,n+1}}\le \eta.
\]
Rearranging gives
\[
2^{\,n+1}\ge \frac{b_0-a_0}{\eta}
\quad\Longrightarrow\quad
n \ge \log_2\!\left(\frac{b_0-a_0}{\eta}\right)-1.
\]
Equivalently, in natural logarithms,
\[
\boxed{\,n \ge \frac{\log(b_0-a_0)-\log\eta}{\log 2} - 1.}
\]
This inequality is the analogue of the bound in problem II but written
for an absolute-error tolerance \(\eta\).

2. \emph{Why relative error is problematic.} 

In problem II we used a
relative-error criterion \(|r-m_n|/|r|\le\epsilon\), and the derivation
relied on a positive lower bound for \(|r|\) (there we used \(r\ge a_0>0\)).
When \(a_0<0<b_0\) such a positive lower bound need not exist: the true
root \(r\) could be very close to \(0\). If \(r\) is near zero, the
relative error \(|r-m_n|/|r|\) can be arbitrarily large even when
\(|r-m_n|\) is small. Thus, \emph{in general} a relative-error requirement
is not appropriate when the interval crosses zero.




\noindent\rule{\textwidth}{.4pt}

\subsubsection*{VIII.}
(*) Consider solving $f(x) = 0$ ($f \in C^{k+1}$) by Newton's method with the starting point $x_0$ being a root of multiplicity $k$. Note that $\alpha$ is a zero of multiplicity $k$ of the function $f$.
\begin{itemize}
    \item How can a multiple root be detected by examining the behavior of the points $(x_n, f(x_n))$?
    \item Prove that if $r$ is a zero of multiplicity $k$ of the function $f$, then quadratic convergence in Newton's iteration will be restored by making this modification:
    \[
    x_{n+1} = x_n - k \frac{f(x_n)}{f'(x_n)}
    \]
\end{itemize}

\textbf{Solution.}
\paragraph{(1) Detecting a multiple root from the points $(x_n,f(x_n))$.}


As we learned in class, when $\alpha$ weren't a zero of multiplicity of the function f,we have the following equation to evaluate the convergence:

\[
e_{n+1} = e_n^2 \frac{f''(\epsilon)}{2f'(x_n)}
\]

Back to our situation,because $\alpha$ is a zero of multiplicity $k$ of the function $f$ and  $f(x) = 0$ ($f \in C^{k+1}$),we can write $f(x)$ into $f(x) = (x - \alpha)^kg(x)$,$g \in C^{1}$

Similarily, we have:

\[
e_{n+1} = e_n - \frac{f(x_n)}{f'(x_n)}= e_n - \frac{e_ng(x_n)}{kg(x_n) + e_ng'(x_n)}=e_n\frac{(k-1)g(x_n)-e_ng'(x_n)}{kg(x_n) + e_ng'(x_n)}=e_n\frac{(k-1)-e_n\frac{g'(x_n)}{g(x_n)}}{k + e_n\frac{g'(x_n)}{g(x_n)}}
\]

Thus, as \( n \to \infty \),
\[
\frac{e_{n+1}}{e_n} \to \frac{k-1}{k} < 1,
\]
so the convergence is only linear.  
Consequently, a root of multiplicity \( k > 1 \) can be detected numerically by observing that the ratio \( \frac{|x_{n+1} - \alpha|}{|x_n - \alpha|} \) tends to the constant \( \frac{k-1}{k} \), i.e. the error decreases slowly and at a constant rate.



\paragraph{(2) Proof that the modified iteration restores quadratic convergence.}


Write $f$ near $r$ in the form
\[
f(x) = (x-r)^k g(x),
\]
with $g\in C^1$ and $g(r)\neq0$. Then
\[
f'(x) = k(x-r)^{k-1} g(x) + (x-r)^k g'(x).
\]

Let $e_n:=x_n-r$. For the modified iteration
\[
x_{n+1} = x_n - k\frac{f(x_n)}{f'(x_n)},
\]
compute the new error $e_{n+1}=x_{n+1}-r$ using the above factorization.
We have
\[
\begin{aligned}
k\frac{f(x_n)}{f'(x_n)}
&= k\frac{e_n^k g(x_n)}{k e_n^{k-1} g(x_n) + e_n^k g'(x_n)} \\
&= e_n\cdot\frac{k g(x_n)}{k g(x_n) + e_n g'(x_n)}.
\end{aligned}
\]
Therefore
\[
\begin{aligned}
e_{n+1}
&= e_n - k\frac{f(x_n)}{f'(x_n)}
= e_n\left(1 - \frac{k g(x_n)}{k g(x_n) + e_n g'(x_n)}\right)\\[4pt]
&= e_n\cdot\frac{e_n g'(x_n)}{k g(x_n) + e_n g'(x_n)}
= e_n^2\cdot\frac{g'(x_n)}{k g(x_n) + e_n g'(x_n)}.
\end{aligned}
\]

Since $g$ is continuous and $g(r)\neq0$, for $x_n$ close to $r$ the
denominator $k g(x_n)+e_n g'(x_n)$ stays close to $k g(r)\neq0$. Thus
there exists a neighbourhood of $r$ where the factor
\[
C_n:=\frac{g'(x_n)}{k g(x_n) + e_n g'(x_n)}
\]
is bounded and tends to the limit $C:=g'(r)/(k g(r))$ as $n\to\infty$.
Hence for $n$ large we get the error law
\[
e_{n+1} = C_n\, e_n^2,
\]
with $C_n\to C\neq0$. This shows the iteration is \emph{quadratically}
convergent (error is proportional to square of previous error) once the
iterates are sufficiently close to the root.

Thus the modification $x_{n+1}=x_n-k\,f(x_n)/f'(x_n)$ restores the usual
quadratic convergence for a root of multiplicity $k$.


\end{document}